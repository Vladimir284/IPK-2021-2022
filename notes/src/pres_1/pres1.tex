\documentclass[10pt,xcolor=pdflatex,hyperref={unicode}]{beamer}
\usepackage{newcent}
\usepackage[utf8]{inputenc}
\usepackage[slovak]{babel}
%\usepackage[T1]{fontenc}
\usepackage{hyperref}
\usepackage{fancyvrb}

% Added for compilation for current project
\usetheme{FIT}

%%%%%%%%%%%%%%%%%%%%%%%%%%%%%%%%%%%%%%%%%%%%%%%%%%%%%%%%%%%%%%%%%%
\title[IPK Zhrnutie]{Úvod}

\author[]{}

\institute[]{Brno University of Technology, Faculty of Information Technology\\
Bo\v{z}et\v{e}chova 1/2. 612 66 Brno - Kr\'alovo Pole\\
login@fit.vutbr.cz}

%\institute[]{Fakulta informačních technologií
%Vysokého učení technického v Brně\\
%Bo\v{z}et\v{e}chova 1/2. 612 66 Brno - Kr\'alovo Pole\\
%login@fit.vutbr.cz}

% Neskor vyberiem
%\date{January 1, 2016}
\date{\today}
%\date{} % bez data / without date


%%%%%%%%%%%%%%%%%%%%%%%%%%%%%%%%%%%%%%%%%%%%%%%%%%%%%%%%%%%%%%%%%%

\begin{document}

    \frame[plain]{\titlepage}

%    \begin{frame}\frametitle{Frame Title}
%    Example \emph{content}.
%    \end{frame}


    \section{Internet}
    \begin{frame}
        \frametitle{WAN}
        \begin{itemize}
            \item Wide area network
            \item Počítačová sieť pokrývajúca rozľahlé geografické územie
        \end{itemize}
    \end{frame}


    \section{Komunikácia}
    \section{Pripojenie k ISP}
    \begin{frame}
        \frametitle{ISP}
        \begin{itemize}
            \item Internet service provider
            \item Poskytovaťeľ internetu :D, Telekom O2, Slovanet, UPC ...
        \end{itemize}
    \end{frame}


    \begin{frame}
        \frametitle{Domáce siete}
        \emph{SOHO}
        \begin{itemize}
            \item Small office, home office
            \item Lacné sieťové zariadenia
        \end{itemize}
        \emph{ADSL}
        \begin{itemize}
            \item Asymmetric Digital Subscriber Line
            \item Nerovnomerné rozdelenie príjmu a odovzdaní dát
        \end{itemize}
        \emph{FTTH}
        \begin{itemize}
            \item Fiber to the home
            \item Optické pripojenie domácnosti k internetu
        \end{itemize}
        \emph{NAT}
        \begin{itemize}
            \item Network address translation
            \item Preklad adries pre presmerovanie toku dát
        \end{itemize}
    \end{frame}

    \begin{frame}
        \frametitle{Modem}
        \begin{itemize}
            \item Modulator-demodulator
            \item Prevádza digitány signál do vyších vrstiev, povoluje pripojenie zariadenia do fzyickej vrstvy
        \end{itemize}
    \end{frame}

    \begin{frame}
        \frametitle{FDM}
        \begin{itemize}
            \item Frequency-division multiplexing
            \item Spojenie viacerých frekvecíí do jednej
        \end{itemize}
    \end{frame}

    \begin{frame}
        \frametitle{LTE}
        \begin{itemize}
            \item Long term evolution
            \item Štandard pre bezdrôtovú, širokopásmovú, dátovú komunikáciu
        \end{itemize}
    \end{frame}

    \begin{frame}
        \frametitle{GSM}
        \begin{itemize}
            \item Global system for mobile comunication
            \item Preklad asi nie je potrebný :)
        \end{itemize}
    \end{frame}


    \section{Domáca sieť}
    \begin{frame}
        \frametitle{Ethernet}
        \begin{itemize}
            \item Spôsob prepojovania viacerých zariadení do lokálnej siete
        \end{itemize}
    \end{frame}

    \begin{frame}
        \frametitle{Wifi}
        \begin{itemize}
            \item Terminus technikus
        \end{itemize}
    \end{frame}

    \begin{frame}
        \frametitle{TDM}
        \begin{itemize}
            \item Time division multiplexing
            \item Spojenie viacerých komunikačných kanálov do jedného rozlíšované na základe časovej synchronizácie
        \end{itemize}
    \end{frame}

    \begin{frame}
        \frametitle{Bonding}
        \begin{itemize}
            \item Spájanie
            \item Spojenie viacerých kanálov do jedného
        \end{itemize}
    \end{frame}

    \begin{frame}
        \frametitle{Paket}
        \begin{itemize}
            \item Malý segment väčšej správy
        \end{itemize}
    \end{frame}


    \section{Štruktúra internetu}
    \begin{frame}
        \emph{NAP}
        \begin{itemize}
            \item Network acces point
            \item Bod pripojenia na internet
        \end{itemize}
        \emph{IXP}
        \begin{itemize}
            \item Interet exchange point
            \item Kde sa stretáva viacero ISP
        \end{itemize}
    \end{frame}

    \section{Oneskorenie, strata a prepustnosť}
    \begin{frame}\frametitle{Model TCP/IP}
    \begin{itemize}
        \item Layer 5 - Application layer - Aplikačná vrstva - Komunikácia medzi aplikáciami
        \item Layer 4 - Transport layer - Transportná vrstva
        \item Layer 3 - Network - Sieťová vrstva
        \item Layer 2 - Link - Linková vrstva
        \item Layer 1 - Physical - Fyzická vrstva - Komunikácia medzi fyzickýi zariadeniami
    \end{itemize}
    \end{frame}


    \bluepage{Thank You For Your Attention !}

\end{document}