\documentclass[10pt,xcolor=pdflatex,hyperref={unicode}]{beamer}
\usepackage{newcent}
\usepackage[utf8]{inputenc}
\usepackage[slovak]{babel}
%\usepackage[T1]{fontenc}
\usepackage{hyperref}
\usepackage{fancyvrb}

% Added for compilation for current project
\usetheme{FIT}

%%%%%%%%%%%%%%%%%%%%%%%%%%%%%%%%%%%%%%%%%%%%%%%%%%%%%%%%%%%%%%%%%%
\title[IPK Zhrnutie]{IPV4}

\author[]{Vladimír Mečiar}

\institute[]{Brno University of Technology, Faculty of Information Technology\\
Bo\v{z}et\v{e}chova 1/2. 612 66 Brno - Kr\'alovo Pole\\
login@fit.vutbr.cz}

%\institute[]{Fakulta informačních technologií
%Vysokého učení technického v Brně\\
%Bo\v{z}et\v{e}chova 1/2. 612 66 Brno - Kr\'alovo Pole\\
%login@fit.vutbr.cz}

% Neskor vyberiem
%\date{January 1, 2016}
\date{\today}
%\date{} % bez data / without date

%%%%%%%%%%%%%%%%%%%%%%%%%%%%%%%%%%%%%%%%%%%%%%%%%%%%%%%%%%%%%%%%%%

\begin{document}

    \frame[plain]{\titlepage}

    \begin{frame}
        \frametitle{TTL}
        \begin{itemize}
            \item Time to live
            \item Dec pri každom odslaní a keď = 0 tak sa paket zahodí
        \end{itemize}
    \end{frame}

    \begin{frame}
        \frametitle{MTU}
        \begin{itemize}
            \item Maximum transision unit
            \item Maximálna veľkosť IP datagramu ktorú vie zariadenie odoslať
        \end{itemize}
        \emph{PMTUD}
        \begin{itemize}
            \item Path MTU discovery
            \item Spôsob ako zistiť po ceste MTU v komunikácií medzi dvoma hosťami s vyhýbaním sa IP fragmentácií
        \end{itemize}
    \end{frame}

    \begin{frame}
        \frametitle{VLSM}
        \begin{itemize}
            \item Variable length subnet mask
        \end{itemize}
    \end{frame}

    \begin{frame}
        \frametitle{CIDR}
        \begin{itemize}
            \item Classles interdomain routing
        \end{itemize}
    \end{frame}

    \begin{frame}
        \frametitle{Správa adresného priestoru}
        Prideľovanie ip priestoru \\
        \emph{RIR}
        \begin{itemize}
            \item Regional Ingernet registries
        \end{itemize}
        \emph{NIR}
        \begin{itemize}
            \item National Internet registries
        \end{itemize}
        \emph{LIR}
        \begin{itemize}
            \item Local Internet registries
        \end{itemize}
        \emph{ISP} \\
        \emph{EU}
        \begin{itemize}
            \item End users
        \end{itemize}
    \end{frame}

    \begin{frame}
        \frametitle{ICANN}
        \begin{itemize}
            \item Internet Corporation for Assigned Names and Numbers
        \end{itemize}
    \end{frame}

    \begin{frame}
        \frametitle{ICMP}
        \begin{itemize}
            \item Internet control message protocol
            \item Používaný hosťami, oznamovanie chýb smerovači
        \end{itemize}
    \end{frame}

    \begin{frame}
        \frametitle{DHCP}
        \begin{itemize}
            \item Dynamic host configuration protocol
            \item Mechanizmus prideľovania IP adries a TCP/IP parametrov
        \end{itemize}

    \end{frame}

\end{document}